% LaTeX resume using res.cls
\documentclass[margin,line,10pt]{res}
\topmargin=-.25in
\oddsidemargin -.5in
\evensidemargin -.5in
\textwidth=6.0in
\itemsep=0in
\parsep=1in
%\usepackage{helvetica} % uses helvetica postscript font (download helvetica.sty)
%\usepackage{newcent}   % uses new century schoolbook postscript font 
\usepackage{mathpazo} % add possibly `sc` and `osf` options
\usepackage{eulervm}
\begin{document}

\name{HAO WU}
% \address used twice to have two lines of address
% IN CHINA
% \address{\small $\bullet$ wu.hao.cz.21@gmail.com $\bullet$ +86 158-9508-0059 $\bullet$ 800 Dongchuan Rd. Shanghai China, 200240}
% IN PA
\address{\small $\bullet$ wu.hao.cz.21@gmail.com $\bullet$ 1 (412)-616-6280 $\bullet$ Apt.603, South Aiken Ave., Pittsburgh, PA, 15232}
 
\begin{resume}

\section{EDUCATION}
{
\small
\textbf{Carnegie Mellon University, Pittsburgh, U.S.} \hfill Aug. 2017 --- May. 2019\\
\textit{Information Networking Institute}\\
Expect M.S. in Information Networking
}\\
{
\small
\textbf{Shanghai Jiao Tong University, Shanghai, China} \hfill Sept. 2013 --- Aug. 2017\\
\textit{University of Michigan - Shanghai Jiao Tong University Joint Institute (UM-SJTU JI)}\\
B.S. in Electrical and Computer Engineering
}


\section{SELECTED\\PROJECTS}
{
\small
{\bf Machine Learning: Think and Speek} \hfill Apr. 2016 --- Dec. 2016\\
{\it \textbf{Team Leader}, Capstone with {\bf Siemens}}
}
\begin{itemize}
\setlength{\itemsep}{0pt}
\setlength{\parskip}{0pt}
\setlength{\parsep}{0pt}
\item {\small Developed intelligent sensor monitoring system for the manufacturing scenario at factory which could collect sensor data on database}
\item {\small Applied neural network models based on collected data to recognize type of component and detect exceptional situations}
\item {\small Report predictions at interactive mobile client which is controlled by voice input}
\end{itemize}
{
\small
{\bf Shallow Discourse Parsing of Implicit Discourse Relations} \hfill Apr. 2016 ---  Aug. 2016\\
{\it \textbf{Team Leader}, SJTU}
}
\begin{itemize}
\setlength{\itemsep}{0pt}
\setlength{\parskip}{0pt}
\setlength{\parsep}{0pt}
\item {\small Applied Maximum Entropy Model and achieved 40.5\% accuracy, same as the world best result in 2009}
\item {\small Introduced word embedding into CNN model and improved accuracy to 46.79\%}
\end{itemize}

%------------------------------------------
%{
%\small
%{\bf Personalized Job Recommendation in XING.com } \hfill Apr. 2016 --- Aug. 2016\\
%{\it \textbf{Team Leader}, Data and Knowledge Management Lab, SJTU \\ACM RecSys Challenge 2016}
%}
%\begin{itemize}
%\setlength{\itemsep}{0pt}
%\setlength{\parskip}{0pt}
%\setlength{\parsep}{0pt}
%\item {\small Applied K-means algorithm to realize user and job clustering}
%\item {\small Built recommendation system model with given user features and predicted job postings that a user would click on for specified users}
%\end{itemize}
%------------------------------------------

{
\small
{\bf 32-bit Pipelined Processor Simulation} \hfill Sept. 2015 --- Dec. 2015\\
{\it \textbf{Team Leader}, UM-SJTU JI}
}
\begin{itemize}
\setlength{\itemsep}{0pt}
\setlength{\parskip}{0pt}
\setlength{\parsep}{0pt}
\item {\small Modelled both single-cycle and pipelined MIPS-Architecture CPU in Verilog}
\item {\small Implemented the pipelined CPU on FPGA board}
\item {\small Showed data in register files with SSDs and LEDs}
\end{itemize}
{
\small
{\bf AFL Football Matches Prediction Modeling} \hfill Aug. 2015 --- Sept. 2015\\
{\it \textbf{Team Leader}, Data \& Knowledge Management Lab, SJTU \\ CIKM Machine Learning Competition 2015, Final Rank (16/33)}
}
\begin{itemize}
\setlength{\itemsep}{0pt}
\setlength{\parskip}{0pt}
\setlength{\parsep}{0pt}
\item {\small Select effective features for Australian Rules Football games}
\item {\small Applied Logistic Regression Model to obtain prediction results}
\item {\small Measured the accuracy of models via cross-validation}\\
\end{itemize}

\section{WORKS\&\\EXPERIENCE}
{\small
{\bf\normalsize Intel Asia-Pacific R\&D Ltd.} \hfill Aug. 2016 --- Dec. 2016 \\
{\it Big Data Team, Software Engineering Intern} 
}
\begin{itemize}
\setlength{\itemsep}{0pt}
\setlength{\parskip}{0pt}
\setlength{\parsep}{0pt}
\item {\small Researched on open-source distributed toolkits and designed corresponding plugins (e.g., developed node management platform for ETCD system)}
\item {\small Built up OpenStack on cluster with Docker containers}
\item {\small Adopted Agile Software Development approach}
\end{itemize}

{
\small
{\bf Teaching Assistant} \\
{\it Course: Introduction to Cryptography, UM-SJTU JI} \hfill Apr. 2017 --- Aug. 2017\\
{\it Course: Introduction to Operating System, UM-SJTU JI} \hfill Sept. 2016 --- Dec. 2016\\
{\it Course: Introduction to Logic Design, UM-SJTU JI} \hfill May. 2016 --- Aug. 2016 \\
{\it Course: Introduction to Computer and Programming, UM-SJTU JI}\hfill Sept. 2015 --- Dec. 2015
}

\section{HONORS\&\\AWARDS}
{
\small
Shanghai Outstanding Graduates \hfill May. 2017\\
Capstone design Silver Award of JI Design Expo \hfill Dec. 2016\\
SJTU Outstanding Undergraduate Scholarship {(Top 10\% in SJTU)}\hfill2014\&2015\&2016\\
Dean's List {(term GPA$>$3.5)} \hfill2014\&2015\&2016\\
Honorable Mention of Mathematical Contest In Modeling (MCM) \hfill2014\\
ACM International Collegiate Programming Contest Jing Hua Division, Silver Medal\hfill 2013\\
}

\section{PERSONAL}
{\small \bf COUMPUTER SKILLS}
\begin{itemize}
\setlength{\itemsep}{0pt}
\setlength{\parskip}{0pt}
\setlength{\parsep}{0pt}
\item {\small \textbf{Platform:} Windows, Linux, OS X}
\item {\small \textbf{Programming Language:} C/C++, Python, Verilog HDL, MIPS Assembly, \LaTeX}
\end{itemize}
{\small \bf LANGUAGE SKILLS}
\begin{itemize}
\setlength{\itemsep}{0pt}
\setlength{\parskip}{0pt}
\setlength{\parsep}{0pt}
\item {\small Mandarin (native), English (fluent)}
\end{itemize}
\iffalse
{\small \bf EXTRA CURRICULAR}
\begin{itemize}
\setlength{\itemsep}{0pt}
\setlength{\parskip}{0pt}
\setlength{\parsep}{0pt}
\item {\small Elected {\it Machine Learning, Stanford University} in Coursera.org}
\end{itemize}
{\small \bf ACTIVITIES}
\begin{itemize}
\setlength{\itemsep}{0pt}
\setlength{\parskip}{0pt}
\setlength{\parsep}{0pt}
\item {\small College Basketball team, UM-SJTU JI, Member}
\item {\small Student Union Department of Publicity, UM-SJTU JI, Member}
\end{itemize}
\fi
\end{resume}
\end{document}